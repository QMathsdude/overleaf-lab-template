\section{Results}
\subsection{Observations}
In this section, display the data that you have obtained.

\begin{table}[!h]
    \centering
    \rowcolors{2}{UM-blue!10}{white}
    \caption{A table without vertical lines.}
    \begin{tabular}{ccccc}
        \toprule
        \tbluebf{Column 1}&\tbluebf{Column 2}&\tbluebf{Column 3}&\tbluebf{Column 4}&\tbluebf{Column 5}\\
        \midrule
            Entry 1&1&2&3&4\\
            Entry 2&1&2&3&4\\
            Entry 3&1&2&3&4\\
            Entry 4&1&2&3&4\\
        \bottomrule
    \end{tabular}
    \label{tab:5.1}
\end{table}

\subsection{Data Analysis}
In this section, display any relevant figures, tables or equations after you have analysed them.

\begin{table}[!h]
    \centering
    \rowcolors{2}{UM-blue!10}{white}
    \caption{A table with vertical lines.}
    \begin{tabular}{c|cccc}
        \toprule
        \tbluebf{Column 1}&\tbluebf{Column 2}&\tbluebf{Column 3}&\tbluebf{Column 4}
        &\tbluebf{Column 5}\\
        \midrule
            Entry 1&1&2&3&4\\
            Entry 2&1&2&3&4\\
            Entry 3&1&2&3&4\\
            Entry 4&1&2&3&4\\
        \bottomrule
    \end{tabular}
    \label{tab:5.2}
\end{table}

\noindent \verb|Block| equations are automatically coloured blue, however \verb|inline| equations are not because it will mess up some paragraph spacing.
\begin{equation}
    i\hbar \frac{\partial}{\partial t} \Psi(\mathbf{r}, t) = \hat{H} \Psi(\mathbf{r}, t)
\end{equation}

\noindent You can also create a blue horizontal line using \verb|\lblue|.

\noindent \lblue

\newpage