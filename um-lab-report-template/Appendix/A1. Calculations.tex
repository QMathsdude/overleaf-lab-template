% Set indentation to zero from this page
\setlength{\parindent}{0pt} 

% Calculations
\section{Calculations}\label{appendix:A}
Show all the calculations you have made for this experiment.
\subsection{Matrix Equation}
Simultaneous equation can be written as:
\begin{equation*}
    A \mathbf{x} = \mathbf{b};\quad \text{where } A =
    \begin{bmatrix}
        a_1 & b_1 & c_1 \\
        a_2 & b_2 & c_2 \\
        a_3 & b_3 & c_3
    \end{bmatrix}, \quad
    \mathbf{x} =
    \begin{bmatrix}
        x \\ y \\ z
    \end{bmatrix}, \quad
    \mathbf{b} =
    \begin{bmatrix}
        b_1 \\ b_2 \\ b_3
    \end{bmatrix}.
\end{equation*}

\noindent To solve for $\mbluebf{x}$, matrix multiply $\mblue{A^{-1}}$ on both sides:
\begin{equation}
    \mathbf{x}=A^{-1}\mathbf{b}.
\end{equation}

\subsection{Kirchhoff's Laws}
Kirchhoff's Voltage Law (\textbf{KVL}) and Current Law (\textbf{KCL)} at the junction:
\begin{equation}
    \sum_{i=1}^{n} V_i = 0,\quad\text{and}\quad \sum_{i=1}^{n} I_i = 0
\end{equation}

\subsection{Branch-Current Analysis Calculations}
Referring to Figure \ref{fig:4.1}, applying \textbf{KCL}:
\begin{equation*}
    I_1+I_2-I_3=0\tag{1}
\end{equation*}
Now applying \textbf{KVL}:
\begin{align*}
    20-V_1-V_3=0\longrightarrow \qty(1.9\,\text{k})I_1+\qty(6.8\,\text{k})I_3&=20 \tag{2}\\
    10-V_2-V_3=0\longrightarrow \qty(4.7\,\text{k})I_2+\qty(6.8\,\text{k})I_3&=10 \tag{3}
\end{align*}
Putting the three equations together, we obtain simultaneous equations:

\begin{equation*}
    \begin{rcases}
        \begin{aligned}[t]
            I_1 + I_2 - I_3 &= 0 \\
            \qty(1.9\,\text{k})I_1+\qty(6.8\,\text{k})I_3&=20 \\
            \qty(4.7\,\text{k})I_2+\qty(6.8\,\text{k})I_3&=10
        \end{aligned}
    \end{rcases}
\end{equation*}

Rewriting the above in matrix form:
\begin{equation*}
    \begin{bmatrix}
        1 & 1 & -1 \\
        1.9 \, \text{k} & 0 & 6.8 \, \text{k} \\
        0 & 4.7 \, \text{k} & 6.8 \, \text{k}
    \end{bmatrix}
    \begin{bmatrix}
        I_1 \\
        I_2 \\
        I_3
    \end{bmatrix}
        =
    \begin{bmatrix}
        0 \\
        20 \\
        10
    \end{bmatrix}
    \equiv A \mathbf{x} = \mathbf{b}
\end{equation*}

\newpage

The inverse of matrix $\mblue{A}$ is:
\begin{equation*}
    A^{-1}=
    \begin{bmatrix}
        \nicefrac{1598}{2173} & \nicefrac{23}{86920} & \nicefrac{-17}{108650} \\
        \nicefrac{340}{2173} & \nicefrac{-17}{108650} & \nicefrac{39}{217300} \\
        \nicefrac{-235}{2173} & \nicefrac{47}{434600} & \nicefrac{1}{43460}
    \end{bmatrix}
\end{equation*}

Therefore, we obtain the value of $\mbluebf{x}$:
\begin{align}
    \mathbf{x}&=A^{-1}\mathbf{b} \nonumber\\
    \mathbf{x}&=
    \begin{bmatrix}
        \nicefrac{1598}{2173} & \nicefrac{23}{86920} & \nicefrac{-17}{108650} \\
        \nicefrac{340}{2173} & \nicefrac{-17}{108650} & \nicefrac{39}{217300} \\
        \nicefrac{-235}{2173} & \nicefrac{47}{434600} & \nicefrac{1}{43460}
    \end{bmatrix}
    \begin{bmatrix}
        0 \\ 20 \\ 10
    \end{bmatrix} \nonumber \\
    \begin{bmatrix}
        I_1 \\ I_2 \\ I_3
    \end{bmatrix}
    & \approx
    \begin{bmatrix}
        3.728\,\text{mA} \\ -1.335\,\text{mA} \\ 2.793\,\text{mA}
    \end{bmatrix}
\end{align}

\newpage